\include{settings}
\usepackage{amsmath}

\begin{document}	% начало документа

% Титульная страница
\include{titlepage}

% Содержание
\renewcommand\contentsname{\centerline{Содержание}}
\tableofcontents
\newpage

\listoftables
\newpage

\section{Постановка задачи}
Сгенерировать выборку объёмом 100 элементов для нормального распределения $N(x, 0, 1)$. По сгенерированной выборке оценить параметры $\mu$ и $\sigma$ нормального закона методом максимального правдоподобия. В качестве основной гипотезы $H_0$ будем считать, что сгенерированное распределение имеет вид $N(x, \widehat{\mu}, \widehat{\sigma})$. Проверить основную гипотезу, используя критерий согласия $\chi^2$. В качестве уровня значимости взять $\alpha=0.05$. Привести таблицу вычислений $\chi^2$.

\section{Теория}
\subsection{Метод максимального правдоподобия}
Пусть $x_1,...,x_n$ — случайная выборка из генеральной совокупности с плотностью вероятности $f(x,\theta)$; $L(x1,...,x_n,\theta)$ — функция правдоподобия (ФП), представляющая собой совместную плотность вероятности независимых с.в. $x_1,...,x_n$ и рассматриваемая как функция неизвестного параметра $\theta$:
\begin{equation} \label{task}
L(x_1,...,x_n,\theta)=f(x_1,\theta)f(x_2,\theta)...f(x_n,\theta)
\end{equation}
\textbf{Определение.} \emph{Оценкой максимального правдоподобия} (о.м.п) будем называть такое значение $\widehat{\theta}_{\text{мп}}$ из множества допустимых значений параметра $\theta$, для которого ФП принимает наибольшее значение при заданных $x_1,...,x_n$:
\begin{equation} \label{task2}
\widehat{\theta}_{\text{мп}}=\text{arg }\max_\theta L(x_1,...,x_n,\theta)
\end{equation}
Если ФП дважды дифференцируема, то её стационарные значения даются корнями уравнения
\begin{equation}
\frac{\partial L(x_1,...,x_n,\theta)}{\partial \theta}=0
\end{equation}
Достаточным условием того, чтобы некоторое стационарное значение $\tilde{\theta}$ было локальным максимумом, является неравенство
\begin{equation}
\frac{\partial^2 L(x_1,...,x_n,\tilde{\theta})}{\partial \theta^2}<0
\end{equation}
Определив точки локальных максимумов ФП (если их несколько), находят наибольший, который и даёт решение задачи (\ref{task}).\\
Часто проще искать максимум логарифма ФП, так как он имеет максимум в одной точке с ФП:
\begin{equation}
\frac{\partial \text{ln}L}{\partial \theta}=\frac{1}{L}\frac{\partial L}{\partial \theta}, \text{ если } L>0,
\end{equation}
и соответственно решать уравнение
\begin{equation}
\frac{\partial \text{ln}L}{\partial \theta}=0,
\end{equation}
которое называют \emph{уравнением правдоподобия}.\\
В задаче оценивания векторного параметра $\theta=(\theta_1,...\theta_m)$ аналогично (\ref{task2}) находится максимум ФП нескольких аргументов:
\begin{equation}
\widehat{\theta}_{\text{мп}}=\text{arg }\max_{\theta_1,...\theta_m} L(x_1,...,x_n,\theta_1,...\theta_m)
\end{equation}
и в случае дифференцируемости ФП выписывается система уравнений правдоподобия
\begin{equation}
\frac{\partial L}{\partial \theta_k}=0 \text{ или } \frac{\partial \text{ln}L}{\partial \theta_k}=0, k=1,...,m.
\end{equation}

\subsection{Проверка гипотезы о законе распределения генеральной совокупности. Метод хи-квадрат}
Исчерпывающей характеристикой изучаемой случайной величины является её закон распределения. Поэтому естественно стремление исследователей построить этот закон приближённо на основе статистических данных.\\

Сначала выдвигается гипотеза о виде закона распределения.\\

После того как выбран вид закона, возникает задача оценивания его параметров и проверки (тестирования) закона в целом.\\

Для проверки гипотезы о законе распределения применяются критерии согласия. Таких критериев существует много. Мы рассмотрим наиболее обоснованный и наиболее часто используемый в практике — критерий $\chi^2$ (хи-квадрат), введённый К.Пирсоном (1900 г.) для случая, когда параметры распределения известны. Этот критерий был существенно уточнён Р.Фишером (1924 г.), когда параметры распределения оцениваются по выборке, используемой для проверки.\\

Мы ограничимся рассмотрением случая одномерного распределения.\\

Итак, выдвинута гипотеза $H_0$ о генеральном законе распределения с функцией распределения $F(x)$.\\

Рассматриваем случай, когда гипотетическая функция распределения $F(x)$ не содержит неизвестных параметров.\\

Разобьём генеральную совокупность, т.е. множество значений изучаемой случайной величины $X$ на $k$ непересекающихся подмножеств $\Delta_1, \Delta_2, ... , \Delta_k$.\\

Пусть $p_i=P(X\in\Delta_i), i=1,...,k$.\\

Если генеральная совокупность — вся вещественная ось, то подмножества $\Delta_i=(a_{i-1}, a_i]$ — полуоткрытые промежутки ($i=2, ... , k-1$). Крайние промежутки будут полубесконечными: $\Delta_i=(-\infty, a_1], \Delta_k=(a_{k-1}, +\infty)$. В этом случае $p_i=F(a_i)-F(a_{i-1})$; $a_0=-\infty, a_k=+\infty (i=1, ... , k)$.\\

Отметим, что $\sum_{i=1}^{k}p_i=1$. Будем предполагать, что все $p_i>0 (i=1, ... , k)$.\\

Пусть, далее, $n_1, n_2, ... , n_k$ — частоты попадания выборочных элементов в подмножества $\Delta_1, \Delta_2, ... , \Delta_k$ соответственно.\\

В случае справедливости гипотезы $H_0$ относительные частоты $n_i/n$ при большом $n$ должны быть близки к вероятностям $p_i (i=1, ... , k)$, поэтому за меру отклонения выборочного распределения от гипотетического с функцией $F(x)$ естественно выбрать величину
\begin{equation}
Z=\sum_{i=1}^{k}c_i(\frac{n_i}{n}-p_i)^2,
\end{equation}
где $c_i$ — какие-нибудь положительные числа (веса). К.Пирсоном в качестве весов выбраны числа $c_i=n/p_i (i=1, ... , k)$. Тогда получается статистика критерия хи-квадрат К.Пирсона
\begin{equation}
\chi^2=\sum_{i=1}^{k}\frac{n}{p_i}(\frac{n_i}{n}-p_i)^2=\sum_{i=1}^{k}\frac{(n_i-np_i)^2}{np_i},
\end{equation}
которая обозначена тем же символом, что и закон распределения хи-квадрат.\\

К.Пирсоном доказана теорема об асимптотическом поведении статистики $\chi^2$, указывающая путь её применения.\\

\textbf{Теорема К.Пирсона}. Статистика критерия $\chi^2$ асимптотически распределена по закону $\chi^2$ с $k-1$ степенями свободы.


\section{Реализация}
Лабораторная работа выполнена с помощью встроенных средств языка программирования R в среде разработки RStudio. Исходный код лабораторной работы приведён в приложении.


\section{Результаты}
\subsection{Проверка гипотезы о законе распределения генеральной совокупности. Метод хи-квадрат}
\subsubsection{Нормальное распределение}
Метод максимального правдоподобия:
\begin{equation}
\widehat{\mu}\approx-0.16
\end{equation}
\begin{equation}
\widehat{\sigma}\approx0.86
\end{equation}

Критерий согласия $\chi^2$:
\begin{table}[h]
	\centering
		\begin{tabular} {|c|c|c|c|c|c|c|}
			\hline
			$i$ & $a_{i-1}, a_i$ & $n_i$ & $p_i$ & $np_i$ & $n_i-np_i$ & $\frac{(n_i-np_i)^2}{np_i}$ \\ \hline
			1 & $-\infty$, -2 & 2 & 0.015 & 1.599 & 0.411 & 0.106 \\ \hline
			2 & -2, -1.5 & 8 & 0.043 & 4.319 & 3.680 & 3.136 \\ \hline
			3 & -1.5, -1 & 11 & 0.105 & 10.499 & 0.500 & 0.023 \\ \hline
			4 & -1, -0.5 & 12 & 0.183 & 18.303 & -6.303 & 2.171 \\ \hline
			5 & -0.5, 0 & 24 & 0.229 & 22.885 & 1.114 & 0.054 \\ \hline
			6 & 0, 0.5 & 20 & 0.205 & 20.524 & -0.524 & 0.013 \\ \hline
			7 & 0.5, 1 & 12 & 0.132 & 13.203 & -1.203 & 0.109 \\ \hline
			8 & 1, 1.5 & 7 & 0.061 & 6.091 & 0.908 & 0.135 \\ \hline
			9 & 1.5, 2 & 3 & 0.020 & 2.014 & 0.985 & 0.481 \\ \hline
			10 & 2, $+\infty$ & 1 & 0.006 & 0.569 & 0.430 & 0.325 \\ \hline
			$\sum$ & - & 100 & 1 & 100 & 0 & $\chi_B^2$=6.558 \\ \hline
		\end{tabular}
		\caption{Вычисление $\chi^2$ при проверке гипотезы $H_0$ о нормальном законе распределения $N(x, \widehat{\mu}, \widehat{\sigma})$}
\end{table}

Количество промежутков $k=10$.\\

Уровень значимости $\alpha=0.05$.\\

Тогда квантиль $\chi_{1-\alpha}^2(k-1)=\chi_{0.95}^2(9)\approx16.918$.\\

Сравнивая $\chi_B^2=6.558$ и $\chi_{0.95}^2(9)\approx16.918$, видим, что $\chi_B^2<\chi_{0.95}^2(9)$.

\subsubsection{Распределение Лапласа}
Метод максимального правдоподобия:
\begin{equation}
\widehat{\mu}\approx-0.35
\end{equation}
\begin{equation}
\widehat{\sigma}\approx2.92
\end{equation}

Критерий согласия $\chi^2$:
\begin{table}[h]
	\centering
		\begin{tabular} {|c|c|c|c|c|c|c|}
			\hline
			$i$ & $a_{i-1}, a_i$ & $n_i$ & $p_i$ & $np_i$ & $n_i-np_i$ & $\frac{(n_i-np_i)^2}{np_i}$ \\ \hline
			1 & $-\infty$, -4 & 1 & 0.10 & 3.20 & -2.20 & 1.51 \\ \hline
			2 & -4, -2 & 2 & 0.18 & 5.42 & -3.42 & 2.15 \\ \hline
			3 & -2, 0 & 14 & 0.26 & 7.83 & 6.16 & 4.85 \\ \hline
			4 & 0, 2 & 11 & 0.24 & 7.22 & 3.77 & 1.96 \\ \hline
			5 & 2, $+\infty$ & 2 & 0.21 & 6.31 & -4.31 & 2.94 \\ \hline
			$\sum$ & - & 30 & 1 & 30 & 0 & $\chi_B^2$=13.44 \\ \hline
		\end{tabular}
		\caption{Вычисление $\chi^2$ при проверке гипотезы $H_1$ о нормальности закона распределения Лапласа}
\end{table}

Количество промежутков $k=5$.\\

Уровень значимости $\alpha=0.05$.\\

Тогда квантиль $\chi_{1-\alpha}^2(k-1)=\chi_{0.95}^2(4)\approx9.49$.\\

Сравнивая $\chi_B^2=13.44$ и $\chi_{0.95}^2(4)\approx9.49$, видим, что $\chi_B^2>\chi_{0.95}^2(4)$.

\section{Обсуждение}
Заключаем, что гипотеза $H_0$ о нормальном законе распределения $N(x, \widehat{\mu}, \widehat{\sigma})$, на уровне значимости $\alpha=0.05$, согласуется с выборкой.\\
Гипотеза о нормальности закона распределения Лапласа не согласуется с полученной выборкой.

\section{Приложения}
Репозиторий на Github с кодом лабораторной работы:\\
\url{https://github.com/VsevolodMelnikov/Math_Stat/tree/master/lab7}

\end{document}