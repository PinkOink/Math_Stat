\include{settings}
\usepackage{amsmath}

\begin{document}	% начало документа

% Титульная страница
\include{titlepage}

% Содержание
\renewcommand\contentsname{\centerline{Содержание}}
\tableofcontents
\newpage

\listoftables
\newpage

\listoffigures
\newpage

\section{Постановка задачи}
Даны данные из Африки и Арктики, полученные в результате флуоресцентной спектроскопии. Это матрицы эмиссии. Они имеют участки высокой интенсивности, возникающие в результате излучения Релея. \\
Необходимо программно вырезать эти участки, посчитать интегралы по заданным зонам в разных пробах и составить гистограммы интенсивностей отдельно для проб Африки и Арктики.

\section{Теория}
\subsection{Компоненты флуоресценции}
Заданны следующие группы:
\begin{table}[!ht]
	\centering
		\begin{tabular} {|c|c|c|c|}
			\hline
			$E_{x_max}(nm)$ & $E_{m_max}(nm)$ & Тип компонента & Буквенное обозначение \\ \hline
			 320-350 & 420-480 & Humic-like & C \\ \hline
			 250-260 & 380-480 & Humic-like & A \\ \hline
			 310-320 & 380-420 & Marine Humic-like & M \\ \hline
			 270-280 & 300-320 & Tyrosine-like, Protein-like & B \\ \hline
			 270-280 & 320-350 & Tryptophane-like, Protein-like or phenol-like & T \\ \hline
		\end{tabular}
		\caption{Компоненты флуоресценции}
\end{table}

\section{Реализация}
Работа выполнена с помощью встроенных средств языка программирования R в среде разработки RStudio. Для работы с данными использовалась библиотека EEM. Исходный код работы приведён в приложении.\\
\\
Рассмотрим обработку данных на примере проб 1701 из Арктики и $1.1\_70$ из Африки. Визуально их можно представить так:
\begin{figure}[!htb]
    \centering
    \includegraphics[width=0.5\textwidth]{fig/Arctic_1.png}
    \caption{Арктика. Проба 1701}
\end{figure}
\begin{figure}[!htb]
    \centering
    \includegraphics[width=0.5\textwidth]{fig/Africa_1.png}
    \caption{Африка. Проба $1.1\_70$}
\end{figure}
\newpage
Обработаем их с помощью метода delScattering2 библиотеки EEM. Получим:
\begin{figure}[!htb]
    \centering
    \includegraphics[width=0.5\textwidth]{fig/Arctic_2.png}
    \caption{Арктика. Проба 1701}
\end{figure}
\begin{figure}[!htb]
    \centering
    \includegraphics[width=0.5\textwidth]{fig/Africa_2.png}
    \caption{Африка. Проба $1.1\_70$}
\end{figure}
\newpage
Для получения компонент флуоресценции используем метод cutEEM. Полученные участки можно представить так:
 \begin{figure}[!htb]
    \centering
    \includegraphics[width=0.5\textwidth]{fig/Arctic_3_C.png}
    \caption{Арктика. Проба 1701. C}
\end{figure}
 \begin{figure}[!htb]
    \centering
    \includegraphics[width=0.5\textwidth]{fig/Arctic_3_A.png}
    \caption{Арктика. Проба 1701. A}
\end{figure}
 \begin{figure}[!htb]
    \centering
    \includegraphics[width=0.5\textwidth]{fig/Arctic_3_M.png}
    \caption{Арктика. Проба 1701. M}
\end{figure}
\newpage
 \begin{figure}[!htb]
    \centering
    \includegraphics[width=0.5\textwidth]{fig/Arctic_3_B.png}
    \caption{Арктика. Проба 1701. B}
\end{figure}
 \begin{figure}[!htb]
    \centering
    \includegraphics[width=0.5\textwidth]{fig/Arctic_3_T.png}
    \caption{Арктика. Проба 1701. T}
\end{figure}

 \begin{figure}[!htb]
    \centering
    \includegraphics[width=0.5\textwidth]{fig/Africa_3_C.png}
    \caption{Африка. Проба $1.1\_70$. C}
\end{figure}
\newpage
 \begin{figure}[!htb]
    \centering
    \includegraphics[width=0.5\textwidth]{fig/Africa_3_A.png}
    \caption{Африка. Проба $1.1\_70$. A}
\end{figure}
 \begin{figure}[!htb]
    \centering
    \includegraphics[width=0.5\textwidth]{fig/Africa_3_M.png}
    \caption{Африка. Проба $1.1\_70$. M}
\end{figure}
 \begin{figure}[!htb]
    \centering
    \includegraphics[width=0.5\textwidth]{fig/Africa_3_B.png}
    \caption{Африка. Проба $1.1\_70$. B}
\end{figure}
\newpage
 \begin{figure}[!htb]
    \centering
    \includegraphics[width=0.5\textwidth]{fig/Africa_3_T.png}
    \caption{Африка. Проба $1.1\_70$. T}
\end{figure}
Значения интегралов по группам для этих 2 проб представлены в таблице:
\begin{table}[!ht]
	\centering
		\begin{tabular} {|c|c|c|c|c|c|}
			\hline
			Проба & C & A & M & B & T \\ \hline
			Арктика. 1701 & 254511 & 106220 & 53529 & 7156 & 12151 \\ \hline 
			Африка. $1.1\_70$ & 27637 & 15902 & 8280 & 8848 & 18995 \\ \hline 
		\end{tabular}
		\caption{Арктика 1701. Африка $1.1\_70$}
\end{table}

\section{Результаты}
Полученные данные были почищены, по заданным группам для каждой пробы были посчитаны интегралы. Результаты представлены в гистограммах.

\begin{figure}[!htb]
    \centering
    \includegraphics[width=0.5\textwidth]{fig/Africa_C.png}
    \caption{Африка. Группа C}
\end{figure}
\newpage
\begin{figure}[!htb]
    \centering
    \includegraphics[width=0.5\textwidth]{fig/Africa_A.png}
    \caption{Африка. Группа A}
\end{figure}

\begin{figure}[!htb]
    \centering
    \includegraphics[width=0.5\textwidth]{fig/Africa_M.png}
    \caption{Африка. Группа M}
\end{figure}
\newpage
\begin{figure}[!htb]
    \centering
    \includegraphics[width=0.5\textwidth]{fig/Africa_B.png}
    \caption{Африка. Группа B}
\end{figure}

\begin{figure}[!htb]
    \centering
    \includegraphics[width=0.5\textwidth]{fig/Africa_T.png}
    \caption{Африка. Группа T}
\end{figure}
\newpage
\begin{figure}[!htb]
    \centering
    \includegraphics[width=0.5\textwidth]{fig/Arctic_C.png}
    \caption{Арктика. Группа C}
\end{figure}

\begin{figure}[!htb]
    \centering
    \includegraphics[width=0.5\textwidth]{fig/Arctic_A.png}
    \caption{Арктика. Группа A}
\end{figure}
\newpage
\begin{figure}[!htb]
    \centering
    \includegraphics[width=0.5\textwidth]{fig/Arctic_M.png}
    \caption{Арктика. Группа M}
\end{figure}

\begin{figure}[!htb]
    \centering
    \includegraphics[width=0.5\textwidth]{fig/Arctic_B.png}
    \caption{Арктика. Группа B}
\end{figure}
\newpage
\begin{figure}[!htb]
    \centering
    \includegraphics[width=0.5\textwidth]{fig/Arctic_T.png}
    \caption{Арктика. Группа T}
\end{figure}


\section{Приложения}
Репозиторий на Github с кодом лабораторной работы:\\
\url{https://github.com/VsevolodMelnikov/Math_Stat/tree/master/project}

\end{document}