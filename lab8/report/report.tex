\documentclass[a4paper,12pt]{extarticle}
\usepackage[utf8x]{inputenc}
\usepackage[T1,T2A]{fontenc}
\usepackage[russian]{babel}
\usepackage[hidelinks]{hyperref}
\usepackage{indentfirst}
\usepackage{listings}
\usepackage{color}
\usepackage{here}
\usepackage{array}
\usepackage{multirow}
\usepackage{graphicx}
\usepackage{subcaption} 
\usepackage{mathtools}

\usepackage{caption}
\renewcommand{\lstlistingname}{Программа} % заголовок листингов кода

\bibliographystyle{ugost2008ls}

\usepackage{listings}
\lstset{ %
extendedchars=\true,
keepspaces=true,
language=C,						% choose the language of the code
basicstyle=\footnotesize,		% the size of the fonts that are used for the code
numbers=left,					% where to put the line-numbers
numberstyle=\footnotesize,		% the size of the fonts that are used for the line-numbers
stepnumber=1,					% the step between two line-numbers. If it is 1 each line will be numbered
numbersep=5pt,					% how far the line-numbers are from the code
backgroundcolor=\color{white},	% choose the background color. You must add \usepackage{color}
showspaces=false				% show spaces adding particular underscores
showstringspaces=false,			% underline spaces within strings
showtabs=false,					% show tabs within strings adding particular underscores
frame=single,           		% adds a frame around the code
tabsize=2,						% sets default tabsize to 2 spaces
captionpos=t,					% sets the caption-position to top
breaklines=true,				% sets automatic line breaking
breakatwhitespace=false,		% sets if automatic breaks should only happen at whitespace
escapeinside={\%*}{*)},			% if you want to add a comment within your code
postbreak=\raisebox{0ex}[0ex][0ex]{\ensuremath{\color{red}\hookrightarrow\space}},
texcl=true,
inputpath=listings,                     % директория с листингами
}

\usepackage[left=2cm,right=2cm,
top=2cm,bottom=2cm,bindingoffset=0cm]{geometry}

%% Нумерация картинок по секциям
\usepackage{chngcntr}
\counterwithin{figure}{section}
\counterwithin{table}{section}

%%Точки нумерации заголовков
\usepackage{titlesec}
\titlelabel{\thetitle.\quad}
\usepackage[dotinlabels]{titletoc}

%% Оформления подписи рисунка
\addto\captionsrussian{\renewcommand{\figurename}{Рисунок}}
\captionsetup[figure]{labelsep = period}

%% Подпись таблицы
%\DeclareCaptionFormat{hfillstart}{\hfill#1#2#3\par}
%\captionsetup[table]{format=hfillstart,labelsep=newline,justification=centering,skip=-10pt,textfont=bf}

%% Путь к каталогу с рисунками
\graphicspath{{fig/}}

%% Внесение titlepage в учёт счётчика страниц
\makeatletter
\renewenvironment{titlepage} {
 \thispagestyle{empty}
}
\makeatother

\DeclarePairedDelimiter\abs{\lvert}{\rvert}%
\DeclarePairedDelimiter\norm{\lVert}{\rVert}%

\usepackage{amsmath}

\begin{document}	% начало документа

% Титульная страница
\begin{titlepage}	% начало титульной страницы

	\begin{center}		% выравнивание по центру

		\large Санкт-Петербургский политехнический университет Петра Великого\\
		\large Институт прикладной математики и механики \\
		\large Кафедра <<Прикладная математика>>\\[6cm]
		% название института, затем отступ 6см
		
		\huge Математическая статистика\\[0.5cm] % название работы, затем отступ 0,5см
		%\huge Методы оптимизации\\[0.5cm] % название работы, затем отступ 0,5см
		\large \textbf{Отчет по лабораторной работе №4}\\[5.1cm]
		%\large \textbf{Отчет по лабораторной работе \\``Решение задач одномерной минимизации ``}\\[5.1cm]
		%\\[5cm]

	\end{center}


	\begin{flushright} % выравнивание по правому краю
		\begin{minipage}{0.25\textwidth} % врезка в половину ширины текста
			\begin{flushleft} % выровнять её содержимое по левому краю

				\large\textbf{Работу выполнил:}\\
				\large Колесник В.Н.\\
				\large {Группа:} 3630102/70201\\
				
				\large \textbf{Преподаватель:}\\
				\large к.ф.-м.н., доцент\\
				\large Баженов Александр Николаевич
				%\large Родионова Елена Александровна

			\end{flushleft}
		\end{minipage}
	\end{flushright}
	
	\vfill % заполнить всё доступное ниже пространство

	\begin{center}
	\large Санкт-Петербург\\
	\large \the\year % вывести дату
	\end{center} % закончить выравнивание по центру

\end{titlepage} % конец титульной страницы

\vfill % заполнить всё доступное ниже пространство


% Содержание
\renewcommand\contentsname{\centerline{Содержание}}
\tableofcontents
\newpage

\listoftables
\newpage


\section{Постановка задачи}
Для двух выборок размерами 20 и 100 элементов, сгенерированных согласно нормальному закону $N(x, 0, 1)$, для параметров положения и масштаба построить асимптотически нормальные интервальные оценки на основе точечных оценок метода максимального правдоподобия и классические интервальные оценки на основе статистик $\chi^2$ и Стьюдента. В качестве параметра надёжности взять $\gamma=0.95$.


\section{Теория}

\subsection{Доверительные интервалы для параметров нормального распределения}

\subsubsection{Доверительный интервал для математического ожидания $m$ нормального распределения}
Дана выборка $(x_1, x_2, ... , x_n)$ объёма $n$ из нормальной генеральной совокупности. На её основе строим выборочное среднее $\overline{x}$ и выборочное среднее квадратическое отклонение$s$𝑠. Параметры $m$ и $\sigma$ нормального распределения неизвестны.\\
\\
Доказано, что случайная величина
\begin{equation}
T=\sqrt{n-1}\frac{\overline{x}-m}{s},
\end{equation}
называемая статистикой Стьюдента, распределена по закону Стьюдента с $n-1$ степенями свободы. \\
Пусть $f_T(x)$ — плотность вероятности этого распределения. \\
Пусть $t_{1-\alpha/2}(n-1)$ — квантиль распределения Стьюдента с $n-1$ степенями свободы и порядка $1-\alpha/2$. \\
Тогда доверительный интервал для $m$ с доверительной вероятностью $\gamma=1-\alpha$ можно получить из равенства:
\begin{equation}
P\begin{pmatrix}\overline{x}-\frac{st_{1-\alpha/2}(n-1)}{\sqrt{n-1}}<m<\overline{x}+\frac{st_{1-\alpha/2}(n-1)}{\sqrt{n-1}}\end{pmatrix}=1-\alpha.
\end{equation}

\subsubsection{Доверительный интервал для среднего квадратического отклонения $\sigma$ нормального распределения}
Дана выборка $(x_1, x_2, ... , x_n)$ объёма $n$ из нормальной генеральной совокупности. На её основе строим выборочную дисперсию $s^2$. Параметры $m$ и $\sigma$ нормального распределения неизвестны. Доказано, что случайная величина $ns^2/\sigma^2$ распределена по закону $\chi^2$ с $n-1$ степенями свободы. \\
Тогда доверительный интервал для $\sigma$ с доверительной вероятностью $\gamma=1-\alpha$ можно получить из равенства:
\begin{equation}
P\begin{pmatrix}\frac{s\sqrt{n}}{\sqrt{\chi_{1-\alpha/2}^2 (n-1)}}<\sigma<\frac{s\sqrt{n}}{\sqrt{\chi_{\alpha/2}^2 (n-1)}}\end{pmatrix}=1-\alpha.
\end{equation}

\subsection{Доверительные интервалы для математического ожидания $m$ и среднего квадратического отклонения $\sigma$ произвольного распределения при большом объёме выборки. Асимптотический подход}
При большом объёме выборки для построения доверительных интервалов может быть использован асимптотический метод на основе центральной предельной теоремы.

\subsubsection{Доверительный интервал для математического ожидания $m$ произвольной генеральной совокупности при большом объёме выборки}
Выборочное среднее $\overline{x}=\sum_{i=1}^n \frac{x_i}{n}$ при большом объёме выборки является суммой большого числа взаимно независимых одинаково распределённых случайных величин. Предполагаем, что исследуемое генеральное распределение имеет конечные математическое ожидание $m$ и дисперсию $\sigma^2$. \\
Пусть $u_{1-\alpha/2}$ — квантиль нормального распределения $N(x, 0, 1)$ порядка $1-\alpha/2$. \\
Тогда доверительный интервал для $m$ с доверительной вероятностью $\gamma=1-\alpha$ можно получить из равенства:
\begin{equation}
P\begin{pmatrix}\overline{x}-\frac{su_{1-\alpha/2}}{\sqrt{n}}<m<\overline{x}+\frac{su_{1-\alpha/2}}{\sqrt{n}}\end{pmatrix}\approx\gamma,
\end{equation}

\subsubsection{Доверительный интервал для среднего квадратического отклонения $\sigma$ произвольной генеральной совокупности при большом объёме выборки}
Выборочная дисперсия $s^2=\sum_{i=1}^n \frac{(x_i-\overline{x})^2}{n}$ при большом объёме выборки является суммой большого числа практически взаимно независимых случайных величин (имеется одна связь $\sum_{i=1}^n x_i=n\overline{x}$, которой при большом $n$ можно пренебречь). Предполагаем, что исследуемая генеральная совокупность имеет конечные первые четыре момента. \\
Пусть $m_4= \frac{1}{n} \sum_{i=1}^n (x_i=\overline{x})^4$ — четвёртый выборочный центральный момент. Тогда доверительный интервал для $\sigma$ с доверительной вероятностью $\gamma=1-\alpha$ можно получить из равенства
\begin{equation}
s(1+U)^{-1/2}<\sigma<s(1-U)^{-1/2},
\end{equation}
где $U=u_{1-\alpha/2}\sqrt{(e+2)/n}$, или равенства
\begin{equation}
s(1-0.5U)<\sigma<s(1+0.5U).
\end{equation} 

\section{Реализация}
Лабораторная работа выполнена с помощью встроенных средств языка программирования R в среде разработки RStudio. Исходный код лабораторной работы приведён в приложении.

\section{Результаты}
\subsection{Доверительные интервалы для параметров нормального распределения}
Значения выборочного среднего и выборочного среднего квадратического равны:
\begin{itemize}
	\item $m=0.03$, $\sigma=0.99$ при $n=20$
	\item $m=0.12$, $\sigma=1.03$ при $n=100$
\end{itemize}
\begin{table}[!ht]
	\centering
		\begin{tabular} {|c|c|c|}
			\hline
			$n=20$ & $m$ & $\sigma$ \\ \hline
			 & $-0.44<m<0.47$ & $0.77<\sigma<1.48$ \\ \hline
			 & & \\ \hline
			$n=100$ & $m$ & $\sigma$ \\ \hline
			 & $-0.08<m<0.32$ & $0.91<\sigma<1.20$ \\ \hline
		\end{tabular}
		\caption{Доверительные интервалы для параметров нормального распределения}
\end{table}

\subsection{Доверительные интервалы для параметров произвольного распределения. Асимптотический подход}
Значения выборочного среднего и выборочного среднего квадратического равны:
\begin{itemize}
	\item $m=0.03$, $\sigma=0.99$ при $n=20$
	\item $m=0.12$, $\sigma=1.03$ при $n=100$
\end{itemize}
\begin{table}[!ht]
	\centering
		\begin{tabular} {|c|c|c|}
			\hline
			$n=20$ & $m$ & $\sigma$ \\ \hline
			 & $-0.40<m<0.47$ & $0.44<\sigma<1.53$ \\ \hline
			 & & \\ \hline
			$n=100$ & $m$ & $\sigma$ \\ \hline
			 & $-0.08<m<0.32$ & $0.79<\sigma<1.27$ \\ \hline
		\end{tabular}
		\caption{Доверительные интервалы для параметров произвольного распределения. Асимптотический подход}
\end{table}

\section{Обсуждение}
\subsection{Доверительные интервалы для параметров распределения}
\begin{itemize}
	\item Генеральные характеристики ($m=0$ и $\sigma=1$) накрываются построенными доверительными интервалами
	\item Доверительные интервалы, полученные по большей выборке, являются соответственно более точными, т.е. меньшими по длине
	\item Доверительные интервалы для параметров нормального распределения более надёжны (меньше по длине), так как основаны на точном, а не асимптотическом распределении
\end{itemize}

\section{Приложения}
Репозиторий на Github с кодом лабораторной работы:\\
\url{https://github.com/VsevolodMelnikov/Math_Stat/tree/lab8}

\end{document}